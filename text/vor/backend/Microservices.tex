\subsection{Мікросервісна архітектура додатків} \label{subsec:microservices}

Монолітна архітектура передбачає реалізацію всіх сервісів ресурсу як єдиної програмної системи. Тобто всі сервіси реалізовані за допомогою одного набору технологій (і мови програмування) і використовують загальні бібліотеки коду. Всі сервіси працюють з одним сервером баз даних.

Мікросервіси є сучасною концепцією реалізації сервісів для систем, що розвиваються. Мікросервісна архітектура полягає в створенні для кожного з логічно відокремлених компонентів системи окремого модулю, пов'язаного з рештою. Часто сервіси групують, якщо вони реалізують схожий, або тісно пов'язаний функціонал.  

Один з принципів проектування мікросервісних додатків додатків визначає, що розмір одного сервісу повинен бути таким, щоб повністю «вміщуватися» в голову програміста \cite{приходченко2018обґрунтування}.

В рамках системи закладено низку модулів, частина з яких використовує у своїй роботі доступ до сервісів Google, зокрема Google Sheets та Google Calendar. При цьому для взаємодії посередництвом Google API потрібно пройти процедуру аутентифікації (рис.~\ref{fig:GoogleServicesAuth}), закладену в методи бібліотек для основних платформ, в тому числі Node.js. Всі пакети мають відкритий вихідний код та поширюються разом з документацією.

\addCodeAsImg{% Auth sequence uml diagram

\documentclass[a4paper,14pt]{article}
\usepackage[english]{babel}
\usepackage[left=3cm, right=1cm, top=1cm, bottom=1cm]{geometry}

\usepackage{tikz-uml}

\sloppy
\hyphenpenalty 10000000

\begin{document}
\thispagestyle{empty}
\begin{center}
\begin{tikzpicture}
\begin{umlseqdiag}
	\umlactor[no ddots, x=1]{User}
	\umlboundary[no ddots, x=5]{App}
	\umldatabase[no ddots, x=11.5, fill=blue!20]{FileSystem}
	\umlboundary[no ddots, x=13.5]{AuthAPI}
	
	\begin{umlcall}[op=auth required action, type=synchron, return=response, padding=-4]{User}{App}
	
		\begin{umlfragment}[type=Auth]
			\begin{umlcall}[op=read credentials, type=synchron, return=credentials]{App}{FileSystem}\end{umlcall}
			\begin{umlcall}[op=authorise, type=synchron, return=oAuth2]{App}{AuthAPI}\end{umlcall}
			
			\begin{umlfragment}[type=oAuth2, label=Error, fill=green!10]
				\umlcreatecall[no ddots, x=8.25]{App}{oAuth2}
				\begin{umlcall}[op=get new token, type=synchron, return=oAuth2]{App}{oAuth2}
					\begin{umlcall}[op=generate URL, type=synchron, return=URL]{oAuth2}{AuthAPI}\end{umlcall}
					\begin{umlcall}[op=prompt interface, type=synchron, return=code]{oAuth2}{App}
						\begin{umlcall}[op=visit URL request, type=synchron, return=input code]{App}{User}\end{umlcall}					
					\end{umlcall}
					\begin{umlcall}[op=set credentials, type=synchron, return=]{oAuth2}{oAuth2}\end{umlcall}
					\begin{umlcall}[op=write token, type=synchron]{oAuth2}{FileSystem}\end{umlcall}	
				\end{umlcall}
				
				\umlfpart[OK]
				\begin{umlcall}[op=create, type=synchron, return=oAuth2]{App}{oAuth2}
					\begin{umlcall}[op=set credentials, type=synchron, return=]{oAuth2}{oAuth2}\end{umlcall}
				\end{umlcall}
				
			\end{umlfragment}
			
		\end{umlfragment}
		
		\begin{umlcall}[op=call action, type=synchron, return=, dt=-1]{App}{App}\end{umlcall}
		
	\end{umlcall}
	
\end{umlseqdiag}
\end{tikzpicture}
\end{center}

\end{document}
}{Авторизація з сервісами Google}{fig:GoogleServicesAuth}

Для компонентів додатку, спроектованого з дотриманням мікросервісної архітектури, справедливі наступні твердження: модулі можна легко замінити, зроблено акцент на незалежність розгортання та оновлення кожного з мікросервісів; модулі організовані навколо функцій, мікросервіс виконує одну елементарну функцію; модулі можуть бути реалізовані з використанням різних мов програмування, виконуватися в різних середовищах, під управлінням різних операційних систем на різних апаратних платформах \cite[159]{кучер2018мікросервісна}.

Загалом, пріоритет віддається на користь найефективнішого для кожної конкретної функції способу розробки і виконання.
