\anonsection{ВИСНОВКИ}

У нас звуки это аналоговый сигнал

То есть непрерывный, зависимость амплитуда-время

мы берем преобразование Фурье

И своими вот этими математиками делаем сигнал дискретным

то есть цифровым

то есть Фурье это тот же аналогово-цифровой преобразователь в некотором роде

обмазались математикой, получили себе зависимость амплитуда-частота
и по определенному правилу (это пока не прониклась) выбираем точку с определенной амплитудой и частотой и считаем что это наша ноту которую мы сыграли

топаем в таблицу с нотами где хэш мапа: {440: "A", 360: "B"}

но не тупо ==

А если значение больше/Меньше этого на условных 20-30 единиц то мы выводим соответственно на экран что это близко к ноте допустим А но пожалуйста подтяните/ослабьте струну чтобы это была именно нота А а не что-то среднее