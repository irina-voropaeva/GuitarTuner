\anonsection{ВИСНОВКИ}

Для виконання поставлених задач було вивчено теоретичний аппарат: основи сольфеджио та нотного запису, основи функціонального аналізу та цифрової обробки сигналів. Для створення проекту було обрано платформу Android. На основі цього було проаналізовано існуючі аналоги додатку, їх плюси та мінуси. 

Було проаналізовано технології, що використовуються для створення мобільних додатків. Також було детально досліджено методи аналізу звукових сигналів; вивчено засоби аналізу звуку у операційній системі Android. 

У розділу (референс на розділ) було розглянено готові рішення для аналізу звукового потіку обраним методом, на основі чого були розроблені вимоги до додатку. Після цього було спроектовано та реалізовано алгоритм швидкого перетворення Фур'є згідно з поставленими вимогами. 

Останнім кроком було вивчено технології тестування та специфіку тестування Android-додатків, був розроблений чек-ліст для мануального тестування додатку у необхідному обсязі а також розроблені автоматичні тести, створені із використанням фреймворку JUnit 4.

Під час розробки додатку у якості системи контролю версій було використано Git із публічним репозиторієм на сайті GitHub (https://github.com/lunmijo/GuitarTuner), що дозволяє використовувати результати проведеного дослідження всім бажаючим під ліцензією MIT.